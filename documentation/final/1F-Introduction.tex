\subsection{Problembeschreibung} % Noah
\begin{frame}
    \frametitle{Problembeschreibung}

    \vspace{0.50cm}

    \textbf{RISC--V:} Offene ISA, die dem Reduced Instruction Set Computer (RISC) Schema folgt.

    \vspace{0.50cm}

%todo single threaded, user space @noah ?
    \textbf{Problem:}
    \begin{itemize}
        \item RISC--V Prozessoren sind noch nicht weit verbreitet.
        \item Entwickler, die Code für RISC--V als Zielplatform kompilieren wollen, können diesen nicht nativ ausführen.
    \end{itemize}

    \vspace{0.50cm}

    \textbf{Lösung:} Emulieren des RISC--V Befehlsatzes auf einem x86--64 Prozessor

    \vspace{0.50cm}

    \begin{block}{Warum x86--64?}
        x86--64 ist der derzeitige Standard für Prozessoren in Laptops und Desktop-PCs.
    \end{block}
\end{frame}

\subsection{RISC--V vs. x86--64} % Noah
\begin{frame}
    \frametitle{RISC--V vs. x86--64}
    \framesubtitle{Gegenüberstellung}

    \begin{minipage}[t]{.47\textwidth}
        \textbf{RISC--V Übersicht:}
        \vspace{0.20cm}
        \begin{itemize}
            \item RISC Schema
            \item Load-Store-Architektur
            \item 31 General Purpose Register
            \item 32 Floating Point Register
            \item 3-Operanden Adressform
            \item Spezielles Zero-Register
        \end{itemize}
    \end{minipage}
    \begin{minipage}[t]{.47\textwidth}
        \textbf{x86--64 Übersicht:}
        \vspace{0.20cm}
        \begin{itemize}
            \item CISC Schema
            \item Register-Memory-Architektur
            \item 16 General Purpose Register
            \item 16 Floating Point (\texttt{XMM}) Register
            \item 2-Operanden Adressform
        \end{itemize}
    \end{minipage}
\end{frame}

\subsection{Dynamische Binärübersetzung} % Noah
\begin{frame}
    \frametitle{Instruction Set Emulation}
    \framesubtitle{Interpretation}

    Das Assembly wird konsekutiv abgearbeitet und jeder Befehl wird einzeln übersetzt.

    \vspace{0.50cm}

    \begin{exampleblock}{Vorteile}
        \begin{itemize}
            \item[$\textcolor{TUMGreen}\blacksquare$] Einfach zu implementieren, portable
            \item[$\textcolor{TUMGreen}\blacksquare$] Keine Erzeugung von JIT Assembler nötig
        \end{itemize}
    \end{exampleblock}

    \vspace{0.50cm}

    \begin{alertblock}{Nachteile}
        \begin{itemize}
            \item[$\textcolor{TUMOrange}\blacksquare$] Geringe Performance
            \item[$\textcolor{TUMOrange}\blacksquare$] Wenig Optimierungspotential (e.g. threaded interpretation)
        \end{itemize}
    \end{alertblock}
\end{frame}

% todo @Noah rework frame titles
\begin{frame}
    \frametitle{Instruction Set Emulation}
    \framesubtitle{Dynamische Binärübersetzung}

    Das Assembly wird schrittweise in die Ziel Architektur übersetzt und dann ausgeführt.

    \vspace{0.50cm}

    \begin{exampleblock}{Vorteile}
        \begin{itemize}
            \item[$\textcolor{TUMGreen}\blacksquare$] Einmaliger Übersetzungsaufwand
            \item[$\textcolor{TUMGreen}\blacksquare$] Höhere Performanze, Optimierung des nativen Codes
        \end{itemize}
    \end{exampleblock}

    \vspace{0.50cm}

    \begin{alertblock}{Nachteile}
        \begin{itemize}
            \item[$\textcolor{TUMOrange}\blacksquare$] Unterstützt keinen selbstverändernden Code
            \item[$\textcolor{TUMOrange}\blacksquare$] Eingeschränkt auf ein Quell- und Zielplatformpaar
        \end{itemize}
    \end{alertblock}
\end{frame}

\begin{frame}
    \frametitle{Instruction Set Emulation}
    \framesubtitle{Statische Binärübersetzung}

    Das gesamten Quellassembly  wird in die Zielarchitektur übersetzt und dann ausgeführt.

    \vspace{0.50cm}

    \begin{exampleblock}{Vorteile}
        \begin{itemize}
            \item[$\textcolor{TUMGreen}\blacksquare$] Einmalige Übersetzung des gesamten Codes, keine Übersetzung zur Laufzeit
            \item[$\textcolor{TUMGreen}\blacksquare$] Hohe Performanz
        \end{itemize}
    \end{exampleblock}

    \vspace{0.50cm}

    \begin{alertblock}{Nachteile}
        \begin{itemize}
            \item[$\textcolor{TUMOrange}\blacksquare$] Code Discorvery Problem
            \item[$\textcolor{TUMOrange}\blacksquare$] Code Location Problem
        \end{itemize}
    \end{alertblock}
\end{frame}