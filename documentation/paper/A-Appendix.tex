\subsection{Download and installation instructions}
The source code for the translator can be downloaded by checking out the project's git repository.
Take care to either \texttt{git clone} the repository with the option flag \texttt{--recursive}, or to run the command
\begin{lstlisting}
	git submodule update --init
\end{lstlisting}
as the repository contains submodules that are required for compilation.

Then, the translator can be built by exeucting
\begin{lstlisting}
	sudo apt-get -y install gcc g++ cmake make autoconf meson
	mkdir build && cd build && cmake .. && make
\end{lstlisting}
in the root directory of the repository.
Note that the build requires CMake version 3.16 or above.
This will build two artifacts:
\begin{description}
	\item[translator] The actual dynamic binary translator.
	For details on the usage, see section \vref{sec:translator-usage}, or execute \texttt{./translator -h}.
	
	\item[test] The unit test binary.
	It can be executed via \texttt{./test} and performs extensive unit testing of the RISC-V instruction implementations, the cache, register file, as well as the parser.
\end{description}

\subsection{Executable program requirements}
We can execute binaries compiled via the tools provided in the RISC-V GNU toolchain\footnote{For further information as well as download and usage instructions, see \url{https://github.com/riscv/riscv-gnu-toolchain} (last accessed on 25.09.2020).}.
The executables need to be linked statically (pass the flag \texttt{-static} to gcc when compiling), as the translator does not support dynamically linked files.

% todo update if floating point comes into play
We currently support binaries compiled for the architecture specifier \texttt{rv64ima}, meaning the compiler is free to utilise the base integer instruction set (\texttt{i}), as well as the instructions provided by the multiplication (\texttt{m}) and atomic standard extensions (\texttt{a}).
This can be achieved by passing \texttt{-march=rv64ima} to gcc\footnote{Note that some architecture strings require recompilation of the toolchain. Also, the aforementioned option implies \texttt{-mabi=lp64}.}.


\subsection{Using the translator}
\label{sec:translator-usage}
% this assumes the state of opt_rewrite, todo update as necessary
\begin{lstlisting}
	./translator [translator option(s)] -f <filename> [guest options]
\end{lstlisting}

Seen above is the syntax for executing the translator with a guest program.
All possible translator options are described in the help text, as seen by executing \texttt{./translator -h}.
Every option after the filename specified via the \texttt{-f} flag is passed along to the guest in its \texttt{argv}, so all options intended for the translator must be passed before \texttt{-f}.

The command line options also include the ability to analyse (\texttt{-a}) the binary to produce a detailed breakdown of which instruction mnemonics and registers the guest will use when executed.
Furthermore, it includes the ability to time the execution of only the guest program by passing the flag \texttt{-b}.

Logging can be controlled by passing the requested category to the \texttt{----log} flag as detailed in the help, and can provide insights into the state of the translator during execution or debugging.
Lastly, it is also possible to selectively disarm optimisation features like the return address stack, block chaining or recursive jump target translation via \texttt{----optimize}.


\subsection{Using the bundled tools}
% todo section about usage of -a, -p, -b and --perf
There seems to be nothing here yet\ldots


\subsection{Version history}
The following mirrors the version control change log of the translator over the course of its development.

\paragraph{v1.2.2}
Continued here...





