\subsection{Download and installation instructions}
The source code for the translator can be downloaded by checking out the project's git repository.
Take care to either \texttt{git clone} the repository with the option flag \texttt{--recursive}, or to run the command
\begin{lstlisting}
	git submodule update --init
\end{lstlisting}
as the repository contains submodules that are required for compilation.

Then, the translator can be built by exeucting
\begin{lstlisting}
	sudo apt-get -y install gcc g++ cmake make autoconf meson
	mkdir build && cd build
	cmake -DCMAKE_BUILD_TYPE=Release \
			-DCMAKE_INTERPROCEDURAL_OPTIMIZATION=true ..
	make
\end{lstlisting}
in the root directory of the repository.
Note that the build requires CMake version 3.15 or above.
This will build two artifacts:
\begin{description}
	\item[translator] The dynamic binary translator.
	For details on the usage, see section~\ref{sec:translator-usage}, or execute \texttt{./translator -h}.
	
	\item[test] The unit test binary.
	It can be executed via \texttt{./test} and performs extensive unit testing of the RISC--V instruction implementations, the cache, register file, as well as the parser.
\end{description}

\subsection{Executable program requirements}
We can execute binaries compiled via the tools provided in the RISC--V GNU toolchain\footnote{For further information as well as download and usage instructions, see \url{https://github.com/riscv/riscv-gnu-toolchain} (last accessed on 25.09.2020).}.
The executables need to be linked statically (pass the flag \texttt{-static} to GCC when compiling), as the translator does not support dynamically linked files.

We currently support binaries compiled for the architecture specifier \texttt{rv64imafd} (also called \texttt{rv64g}), meaning the compiler is free to utilise the base integer instruction set (\texttt{i}), as well as the instructions provided by the multiplication (\texttt{m}), atomic (\texttt{a}) and floating point standard extensions (\texttt{f}, \texttt{d}).
This can be achieved by passing \texttt{--march=rv64g} to GCC\footnote{Note that some architecture strings require recompilation of the toolchain. Also, excluding the floating point instructions in the architecture string implies \texttt{-mabi=lp64}.}.


\subsection{Using the translator}
\label{sec:translator-usage}
\begin{lstlisting}
	./translator [translator option(s)] -f <filename> [guest options]
\end{lstlisting}

Seen above is the syntax for executing the translator with a guest program.
All possible translator options are described in the help text, as seen by executing \texttt{./translator -h}.
Every option after the filename specified via the \texttt{-f} flag is passed along to the guest in its \texttt{argv}, so all options intended for the translator must be passed before \texttt{-f}.

The command line options also include the ability to analyse (\texttt{-a}) the binary to produce a detailed breakdown of which instruction mnemonics and registers the guest will use when executed.
Furthermore, it includes the ability to time the execution of only the guest program (\texttt{-b}) and profile the register and cache usage (\texttt{-p}).

Logging can be controlled by passing the requested category to the \texttt{----log} flag as detailed in the help, and can provide insights into the state of the translator during execution or debugging.
Lastly, it is also possible to selectively disarm optimisation features like the return address stack, block chaining, recursive jump target translation or macro operation fusion via \texttt{----optimize}.


\subsection{Version history}
The following mirrors the version control change log of the translator over the course of its development.

\paragraph{Version 1.3.1 (latest)}
\begin{itemize*}
 	\item fix a bug where the replacement register recency was not reset correctly when loading a non-mapped-register that was already present
 	\item prevent redundant writes to x0 in the A-extension translation
 	\item reorder patterns for more efficient translations
 	\item various minor cleanups
\end{itemize*}


\paragraph{Version 1.3.0}
\begin{itemize*}
 	\item implement support for F-/D-extension including a static register mapping
 	\item expand unit testing coverage to test combinations for floating point instructions
 	\item optimise chaining for conditional branches
 	\item cleanup and fix various small issues in the code base
\end{itemize*}


\paragraph{Version 1.2.4}
\begin{itemize*}
 	\item implement a lazy runtime register replacement strategy, keeping the not-statically-mapped values in the replacement registers as long as possible to prevent redundant memory operations inside blocks
 	\item add logging for static and dynamic register mapping
 	\item implement patterns for MV and LI, ADDI with rs1 == x0 and several shifting combinations as well as zero-extensions via ANDI and 0xff
 	\item use the x86 LEA instruction for faster translations of various input instructions
 	\item rewrite and optimise the return address stack
 	\item fix the \texttt{-m} short option to include \texttt{----optimize=no-fusion}
 	\item use the output of \texttt{git describe} for the version string to include the commit hash in the build
 	\item add inline logging for generated assembly with \texttt{----log=asm-out}
 	\item split up the analyser command line flags to specify what to analyse
 	\item bump C standard to C11
\end{itemize*}


\paragraph{Version 1.2.3}
\begin{itemize*}
 	\item do not page-align the generated code
 	\item implement pattern matching to apply macro operation fusion of multiple RISC--V instructions
 	\item optimise various instructions and clean up legacy code
 	\item gain performance to approx.\ 1.4x--1.7x faster than QEMU
\end{itemize*}


\paragraph{Version 1.2.2}
\begin{itemize*}
 	\item expand the static register mapping for better performance overall
 	\item reallocate the code cache index and rehash all values when it is 50 \% full for better lookup performance on capacity overflow
 	\item rewrite command line options parsing to allow for long options\\(see \texttt{./translator -h})
 	\item allow finer control of specific optimisation features via \texttt{----optimize}
 	\item add instruction pattern matching to the binary analyser to gather data for macro operation fusing
 	\item add a profiler for counting register accesses
 	\item implement emulation for syscalls \texttt{faccessat, getrandom, renameat2}
 	\item remove emulation for the syscall \texttt{clone}
 	\item fix crash when the code cache fills up by increasing the memory space available for translated blocks
\end{itemize*}


\paragraph{Version 1.2.1}
\begin{itemize*}
 	\item add implementations for \texttt{AMOMIN} and \texttt{AMOMAX} instructions
 	\item add extensive unit testing for atomic and arithmetic instructions, as well as the parser
 	\item fix several issues to enable the SPEC CPU 2017 benchmark suite to run
 	\item implement emulation for syscalls \texttt{chdir, pipe2, getdents64, munmap, clone, execve, wait4}
 	\item fix \texttt{ORI} instruction being parsed as \texttt{XORI}
 	\item fix instruction semantics for \texttt{SUB(W)}
 	\item finalize implementation of the return address stack
\end{itemize*}


\paragraph{Version 1.2.0}
\begin{itemize*}
 	\item enable register mapping for translated instructions
 	\item add context switching from host to guest programs
 	\item rework instruction translator function for flexibility
 	\item implement a return address stack
 	\item implement a TLB for cache lookup of blocks
 	\item flip \texttt{-m} translation optimiser flag (enabled by default, flag now turns off optimisations)
\end{itemize*}


\paragraph{Version 1.1.0}
\begin{itemize*}
 	\item cleanup and refactor project files
 	\item remove all C++ usage from translator code
 	\item eliminate standard library usage
 	\item add performance measuring flags to the translator
\end{itemize*}


\paragraph{Version 1.0.1}
\begin{itemize*}
 	\item fix an issue with the \texttt{read} system call that causes blocking problems with gzip
\end{itemize*}


\paragraph{Version 1.0}
The initial release of the translator capable of executing gzip.

This release supports
\begin{itemize*}
 	\item the RISC--V integer instruction set
 	\item the multiplication extension instructions (M)
 	\item the atomic extension instructions (A).
\end{itemize*}

The latter are not yet implemented atomically, however they make the translator compatible with binaries compiled for the architecture \texttt{rv64ima}, with the ABI \texttt{lp64}.



