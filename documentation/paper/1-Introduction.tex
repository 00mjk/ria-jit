The aim of this project is to create an emulator capable of executing code compiled for the RISC-V instruction set architecture on an x86-64 system.

RISC-V is an open ISA first conceptualised in 2010 with the initial goals of research and education in mind.
Its development took the lessons learned in terms of backwards compatibility and future-proofing from other widespread ISAs like Intel x86 into account, and aimed to provide an open interface for the architecture, rather than strict implementation details.
This grants a large freedom to the implementors and greatly increases the flexibility and ease of working with the architecture~\cite[S. 1f]{riscvspec}.

\subsection{Problem description}
Because there is as yet no real hardware available for the RISC-V ISA, developers must rely on emulation in order to execute their software on a foreign architecture.

We aim to provide such an emulator, allowing the execution of RISC-V code on an x86-64 machine by means of dynamic binary translation.

By its very nature, executing code compiled for one architecture on a different one is not an easy task.


\subsection{Comparison of the RISC-V and x86-64 ISAs}
\label{sec:isa-cmp}
%todo (mention pseudoinstructions for later)

Continued here (compare ISAs and note challenges)\ldots













