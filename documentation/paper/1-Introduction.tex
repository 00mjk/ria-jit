RISC-V is an open ISA first conceptualised in 2010 with the initial goals of research and education in mind.
Follows the RISC (Reduced Instruction Set Computer) Scheme (like ARM) in contrast to x86-64\ldots
Its development took the lessons learned in terms of backwards compatibility and future-proofing from other widespread ISAs like Intel x86 into account, and aimed to provide an open interface for the architecture, rather than strict implementation details.
This grants a large freedom to the implementors and greatly increases the flexibility and ease of working with the architecture~\cite[S. 1f]{riscvspec}.
As such it looks to be open to future extensions by already defining a basis for future 128-bit integer instructions and instruction length encodings of up to 176 bits (22 Bytes) already defined and the possibility to expand further.


\subsection{Problem description}
There is already some hardware available for RISC-V (see/maybe cite SIFive), but it is not yet widespread and a lot of developers won't have access to real hardware, so they must rely on emulation to test their code.

We aim to provide such an emulator, allowing the execution of RISC-V code on an x86-64 machine by means of dynamic binary translation.

By its very nature, executing code compiled for one architecture on a different one is not an easy task.


\subsection{Comparison of the RISC-V and x86-64 ISAs}
\label{sec:isa-cmp}
%todo (mention pseudoinstructions for later)

Continued here (compare ISAs and note challenges)\ldots













