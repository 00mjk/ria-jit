The following section aims to provide an in-depth overview of the system's architecture as well as the rationale for major design decisions taken during the implementation.

\subsection{System architecture and execution control flow}
% todo overview, add class diagrams and flow charts

\subsection{Instruction translation process}
% todo

\subsection{Code cache and TLB for block lookup}
% todo

\subsection{Static hybrid register mapping}
% todo register mapping details

\subsection{Optimisation of the generated code}
\subsubsection{Block chaining}
\subsubsection{Recursive jump translation}
% todo optimisation details, maybe split up? more sections?

\subsection{Detailed system call overview}
% todo special description for emulated calls

As described in section \vref{sec:syscall-handling}, we must assume the role of the kernel by handling system calls during the execution of the guest program.
We achieve this by translating the \texttt{ECALL} instruction as a context switch and jump to the \texttt{emulate\_ecall} routine.

As we stored the guest's registers before jumping to the handler, the requested system call index is now available to the DBT in the register file as entry \texttt{a7}.
We may now handle the system calls based on that index and the arguments passed in the registers \texttt{a0} through \texttt{a6}, and write the return value to entry \texttt{a0} of the register file prior to switching the context back to the guest.

We will shortly list the different types of handling required by some system calls that are either not present on the x86-64 host architecture, or may influence or break the state of the DBT.
\begin{description}
	\item[To be continued] \ldots
\end{description}

\begin{table}
	\centering
	\begin{tabular}{ccc}
		\toprule
		\textbf{System Call (index)} & \textbf{Handling} & \textbf{x86-64 base (index)}\\ 
		\midrule
				\texttt{fstatat} (79) & data reformat & \texttt{newfstatat} (262)\\
		\texttt{fstat} (80) & data reformat & \texttt{fstat} (5)\\
		\texttt{exit} (93) & emulate & n/a\\
		\texttt{exit\_group} (94) & emulate & n/a\\
		\texttt{rt\_sigaction} (134) & ignore & n/a\\
		\texttt{brk} (214) & emulate & n/a\\
		\texttt{munmap} (215) & guarded pass-through & \texttt{munmap} (11)\\
		\texttt{mmap} (222) & guarded pass-through & \texttt{mmap} (9)\\
		\bottomrule
	\end{tabular}
	% todo keep up-to-date
	% state: aa55cef4816eb790df21f4742b7cf1f29685da49
	\caption{An overview of the system calls we support that require special handling by the binary translator.}
\end{table}