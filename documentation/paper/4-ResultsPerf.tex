% todo expand this section with further details

\begin{comment}
% ======= EXAMPLE PLOT =======
% Update, copy and modify as needed. Shows tikz usage.
% todo remove before submission (obviously).
% ============================
% The [h] requests TeX put the float <h>ere.
% There are other options for <p>age, <t>op, <b>ottom;
% If it shouldn't float, \usepackage{float} and use [H]
% or a minipage with associated caption.
\begin{figure}[h]
	\centering
	\begin{tikzpicture}
		\begin{axis}[%
			title = {Example bar graph},
			ybar,
			area legend,
			ylabel = {Value in some unit},
			xtick = data,
			xtick style = {draw = none},
			ytick = {0, 2500, 5000, 7500, 10000, 12500},
			scaled y ticks = false,
			symbolic x coords={First category, Second category, Third category},
			ymin = 0, ymax = 12500,
			ymajorgrids = true,
			height = 6.0cm,
			width = \linewidth,
			legend style = {
				at = {(0.5, -0.25)},
				anchor = north,
				legend columns = 3,
				column sep = 0.2cm
			}
		]	
			\addplot+ [
				fill=era-dbt-1,
				draw=black
			] coordinates {
				(First category, 3061)
				(Second category, 10930)
				(Third category, 10971)
			};
			
			\addplot+ [
				fill=era-qemu,
				draw=black
			] coordinates {
				(First category, 3061)
				(Second category, 9092)
				(Third category, 5042)
			};
			
			\addplot+ [
				fill=era-native,
				draw=black
			] coordinates {
				(First category, 4043)
				(Second category, 6092)
				(Third category, 7971)
			};

			
			\legend{First series, Second series, Third series}
		\end{axis}
	\end{tikzpicture}
	\caption{I am an example figure. If you're reading the submitted document, someone forgot to remove me.}
\end{figure}
\end{comment}


Measuring the performance of the DBT was accomplished by using the tools in \textit{SPEC CPU 2017}'s \texttt{intspeed} suite of benchmarks.
This not only generates reproducible and widely accepted results in the industry, it also validates the results produced during the run, thus ruling out any errors in the benchmark's translation.

% todo include benchmark description table (and list of tables) for the spec programs with the benchmark workload description
The \texttt{intspeed} suite also presents a variety of different workloads to the translator that are based on real-life scenarios, thus producing an accurate and understandable overview of the DBT's performance in a non-controlled environment.
Further context is provided by performance testing using the data compression utility \textit{gzip}~\cite{gzip}, where compression time is compared between runs on a native machine, in QEMU and in the DBT.

All testing was performed on an x86--64 8-core \textit{Intel Xeon Bronze 3106} system clocked at $1,70$ GHz base with $78$ GiB of physical memory, running \textit{Ubuntu 18.04.3 LTS}, kernel version \textit{4.15.0-70-generic}.
The DBT was compiled via \texttt{CMAKE\_BUILD\_TYPE} set to \texttt{Release}, which implies \texttt{-O3}.

\subsection{SPEC CPU 2017 Results}
% todo benchmark results


% ======= SPEC CPU Results =======
% Results of the intspeed SPEC CPU 2017 runs.
% todo Keep up to date with performance-improving changes.
% todo normalise the results
% todo update ordering to sorted by ours from bad to best
% ================================
\begin{figure}[h]
	\centering
	\begin{tikzpicture}
		\begin{axis}[%
			title = {SPEC CPU 2017 \texttt{intspeed} Results},
			ybar,
			area legend,
			ylabel = {Execution time [s]},
			xtick = data,
			xtick style = {draw = none},
			xticklabel style = {
				inner sep = 0pt,
				anchor = north east,
				rotate = 60
			},
			ytick = {0,2500,5000,7500,10000},
			scaled y ticks = false,
			symbolic x coords={600.perlbench, 602.gcc, 605.mcf, 620.omnetpp, 623.xalancbmk, 625.x264, 631.deepsjeng, 641.leela, 648.exchange2, 657.xz},
			ymin = 0,
			ymajorgrids = true,
			bar width = 5pt,
			height = 6.0cm,
			width = 0.9\linewidth,
			legend style = {
				at = {(0.015, 0.95)},
				anchor = north west,
				legend columns = 3,
				column sep = 0.2cm
			}
		]	
			% Native results
			\addplot+ [
				fill=era-native,
				draw=black
			] coordinates {
				(600.perlbench, 654)
				(602.gcc, 953)
				(605.mcf, 1409)
				(620.omnetpp, 779)
				(623.xalancbmk, 533)
				(625.x264, 554)
				(631.deepsjeng, 773)
				(641.leela, 1045)
				(648.exchange2, 780)
				(657.xz, 4665)
			};
			
			% QEMU results
			\addplot+ [
				fill=era-qemu,
				draw=black
			] coordinates {
				(600.perlbench, 3061)
				(602.gcc, 3390)
				(605.mcf, 3182)
				(620.omnetpp, 2576)
				(623.xalancbmk, 1711)
				(625.x264, 2921)
				(631.deepsjeng, 2459)
				(641.leela, 3171)
				(648.exchange2, 2213)
				(657.xz, 8915)
			};
			
			% DBT results
			\addplot+ [
				fill=era-dbt-1,
				draw=black
			] coordinates {
				(600.perlbench, 2222)
				(602.gcc, 1956)
				(605.mcf, 2695)
				(620.omnetpp, 1768)
				(623.xalancbmk, 1178)
				(625.x264, 2590)
				(631.deepsjeng, 1603)
				(641.leela, 2080)
				(648.exchange2, 1575)
				(657.xz, 11083)
			};

			\legend{Native, QEMU, DBT}
		\end{axis}
	\end{tikzpicture}
	\caption[SPEC CPU 2017 Results]%
	{Results of \texttt{ref}-workload runs of \textit{SPEC CPU 2017}'s \texttt{intspeed}.}
\end{figure}
% ================================

\subsection{Data compression via gzip}
% gzip results
Next to the results of the \textit{SPEC CPU 2017} suite, it is also valuable to measure the performance of the translator in real-world workloads by running data compression via \textit{gzip}.

The native \textit{gzip} binary for the reference runs was obtained from the default package repositories.
As there is no such executable available for the RISC-V architecture, the sources had to be manually compiled.
% todo Simon? insert info here about compile settings for risc_gzip

% ======= gzip execution time =======
% Execution time of compression (500 MB, 10 runs).
% todo keep up-to-date with performance improvements
% ===================================
\begin{figure}[h]
	\centering
	\begin{tikzpicture}
		\begin{axis}[%
			title = {Random Data Compression (\textit{gzip}, 500 MB, 10 runs)},
			xbar,
			area legend,
			xlabel = {Execution time [s] (lower is better)},
			symbolic y coords = {Native,QEMU,v1.2.2-61},
			xmin = 0, xmax = 145,
			bar shift = 0.0cm,
			y dir = reverse,
			xmajorgrids = true,
			height = 4.5cm,
			width = \linewidth,
			legend style = {
				at = {(0.5, -0.45)},
				anchor = north,
				legend columns = 3,
				column sep = 0.2cm
			}
		]	
			\addplot+ [
				fill=era-native,
				draw=black
			] coordinates {
				(44.14,Native)
			};

			\addplot+ [
				fill=era-qemu,
				draw=black
			] coordinates {
				(121.83,QEMU)
			};
			
			\addplot+ [
				fill=era-dbt-1,
				draw=black
			] coordinates {
				(136.77,v1.2.2-61)
			};
			
			\legend{Native, QEMU, DBT}
		\end{axis}
	\end{tikzpicture}
	\caption[Execution time of gzip compression (500 MB, 10 runs)]%
	{Execution time of gzip file compression (500 MB of random data, 10 runs) in seconds (lower is better).}
	\label{fig:gzip-execution-time}
\end{figure}
% ===================================

Figure \vref{fig:gzip-execution-time} lists the execution times of \textit{gzip} compressing a pseudo-random $500$ MB file sourced from \texttt{/dev/urandom}\footnote{Reproducible via \texttt{base64 /dev/urandom | head -c 524288000 > random.txt;}}.
% todo add analysis as soon as results are finalized

\subsection{Other performance metrics}
By implementing optimisations to the DBT like a return address stack, recursive jump target translation and block chaining as discussed in section \vref{sec:optimise}, we are able to meaningfully increase the performance of the translator in certain workloads, to the point where we outperform \textit{QEMU} significantly.

% ======= merge_sort execution time =======
% Execution time of the merge_sort test.
% todo keep up-to-date with performance improvements
% =========================================
\begin{figure}[h]
	\centering
	\begin{tikzpicture}
		\begin{axis}[%
			title = {Recursive Merge Sort (5000 runs)},
			xbar,
			area legend,
			xlabel = {Execution time [ns] (lower is better)},
			symbolic y coords = {Native,QEMU,DBT unoptimised,DBT optimised},
			xmin = 0,
			bar shift = 0.0cm,
			y dir = reverse,
			xmajorgrids = true,
			height = 4.5cm,
			width = 0.87\linewidth,
			legend style = {
				at = {(0.5, -0.45)},
				anchor = north,
				legend columns = 2,
				column sep = 0.2cm
			}
		]	
			\addplot+ [
				fill=era-native,
				draw=black
			] coordinates {
				(23385.65,Native)
			};

			\addplot+ [
				fill=era-qemu,
				draw=black
			] coordinates {
				(67312.47,QEMU)
			};
			
			\addplot+ [
				fill=era-dbt-2,
				draw=black
			] coordinates {
				(302521.38,DBT unoptimised)
			};
			
			\addplot+ [
				fill=era-dbt-1,
				draw=black
			] coordinates {
				(37565.99,DBT optimised)
			};
			
			\legend{Native, QEMU, DBT (\texttt{----optimize=none}), DBT (optimized)}
		\end{axis}
	\end{tikzpicture}
	\caption[Execution time of merge sort (5000 runs)]%
	{Execution time of merge sort (5000 runs) in nanoseconds (lower is better). Testing performed with v1.2.2-61.}
	\label{fig:dbt-optimisations}
\end{figure}
% =========================================

Figure \vref{fig:dbt-optimisations} shows a manifestation of these performance gains, as seen by executing a benchmark based on a recursive implementation of the merge sort algorithm.






















