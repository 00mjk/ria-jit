In the following, the term \textit{host} will refer to the system of the native architecture the binary translator is built for (in our case, x86--64), and the term \textit{guest} will designate the foreign system we are attempting to emulate (RISC--V).

\subsection{Comparison of the RISC--V and x86--64 ISAs}
\label{sec:isa-cmp}
By its very nature, executing code compiled for one architecture on a different one is not an easy task.
It is obvious that there are major differences in the two architectures, brought forth by RISC--V being a reduced instruction set computer (RISC) architecture and x86--64 a complex instruction set computer (CISC) architecture.

The most relevant distinction between RISC--V and x86--64 for the development of our DBT is the different address format and mismatching numbers of general purpose and floating point registers.
RISC--V's load-store architecture with a three-operand instruction format allows for a more flexible register usage but requires more instructions due to the explicit load/store operations.
x86--64 however is a register-memory architecture with a two-operand instruction format.
Thus, memory accesses are reduced by implicit loads in instructions using a memory operand.

RISC assembler code includes pseudo-instructions that are translated into multiple instructions by the assembler, due to the nature of a reduced instruction set.
One example is the absence of a \texttt{mov} instruction in the RISC--V ISA\@; the assembler translates the pseudo-instruction \texttt{mv} into an \texttt{addi rd, rs1, 0} instruction.

Similarly, the hardware offers no instruction to load a 64 bit immediate into a register, as the fixed-width 32 bit instruction encoding does not allow for it.
Contrary to x86, where this problem would be solved by a single \texttt{mov} instruction, the RISC--V assembler has to build up the immediate in the register by using a specific combination of addition and shifting operations as well as program-counter-relative arithmetic.

In an ideal world, translating a RISC binary for execution on a CISC system would lead to a size reduction of the generated code.
However, in practice, this is nearly impossible.
Efficient fusion of multiple RISC--V instructions into fewer x86--64 instructions (e.g.\ by vectorisation) is difficult even for compilers presented with the entire program context.
When the DBT is only presented with partial block-wise assembly code snippets, some optimisations are impossible to undertake, as there are many unknowable parameters at play.
Those challenges will be further elaborated on in section~\ref{sec:Approach}.

\subsection{Environment setup and memory layout}
\label{sec:memory-layout}
% todo man page citation @Simon
As the DBT is responsible for managing the execution environment of the guest binary in the shared address space, it must also handle the setup of said environment.

The header of the ELF-file (\textit{Executable and Linkable Format}~\cite{elf}) specifies which section(s) of the program need to be loaded, and where in memory they must reside.
The DBT must take care to map the file into memory correctly, while not compromising its own memory region.

Furthermore, the guest registers (see section~\ref{sec:context-switch-reg-handle}) and stack must be initialised in accordance with the architecture specification and calling convention, which necessitates a specific layout of environment and auxiliary parameters as well as command line arguments to be present~\cite[S. 2]{bintrans}.

The stack is set up exactly like it would have been by the linux kernel.
As such, the stack pointer needs to point at the argument count, followed by (towards higher addresses) the zero terminated argument, environment and auxiliary vectors.
Finally some alignment bytes need to be added, so the stack pointer is 16 byte aligned and ABI-compliant.
All of the information can generally be copied from the host in our case.

The memory is laid out as follows:
\begin{table}[h]
	\centering
	\begin{tabular}{rl}
		\toprule
		\textbf{Address range} & \textbf{Usage}\\
		\midrule
		\texttt{0x780000000000}+ & Translator address region\\
		\texttt{0x77ff81000000}+ & JIT generated code\\
		\texttt{0x77ff807fe000}+ & guest stack\\
		\texttt{(last mapped address + 1)}+ & guest heap\\
		\textit{defined by ELF file} & mapped guest binary\\
		\bottomrule
	\end{tabular}
	\caption[Memory layout]%
	{The layout of the memory space.}
	\label{tab:memory-layout}
\end{table}
The translator is linked to a high address at \texttt{0x780000000000} so the range that typical programs use at the bottom of the address space is kept free and guest addresses do not need to be converted and can directly be used.
So the global variables of the translator can be reached with a 32 bit \texttt{RIP}-relative offset from our generated code, the following region is reserved for the translated code.
The lower limit for the guest stack is obtained by using the kernel stack limit and adding a guard page.


\subsection{Partitioning the input code}
Logically, upon facing the task of translation, the DBT must somehow divide the code into chunks it can then process for translation and execution.
The natural choice here is for the translator to partition the code into basic blocks.

\textit{Basic blocks}, by definition, have only a single point of entry and exit;
all other instructions in a single block are executed sequentially and in the order that they appear in the code.
(Of course, this does not take into account mechanisms such as out-of-order execution or system calls as well as interrupt- and exception handling).

So, for our purposes, a basic block will be terminated by any control-flow altering instruction like a jump, call or return statement, or a system call\footnote{These may or may not have control-flow altering effects; they in any case need to be handled this way due to the reasons laid out in section~\ref{sec:syscall-handling}.}.








