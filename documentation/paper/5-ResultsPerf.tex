\subsection{Verification}
The correctness of our translator was checked by extensive unit tests, which were used to check if the translations of a the RISC--V instruction really do perform the operations that are expected.
Using the parameterized tests provided by the Google~Test~\cite{gtest} framework we were able to test a variety of different input values and combinations of statically mapped or not mapped register operands.
Different combinations of using the same register and zero register \texttt{x0} for in- and output are also tested.

Apart from this correctness is also confirmed by being able to sucessfully run the \textit{SPEC CPU 2017} benchmark suite as described in section~\ref{subsec:spec-cpu-2017-benchmark-suite} below.


\subsection{SPEC CPU 2017 Benchmark Suite}
\label{subsec:spec-cpu-2017-benchmark-suite}


\begin{table}
	\centering
	\begin{tabular}{rl}
		\toprule
		\textbf{SPECspeed Benchmark} & \textbf{Workload}\\
		\midrule
		\texttt{600.perlbench} & Perl interpreter\\
		\texttt{602.gcc} & GNU C compiler\\
		\texttt{605.mcf} & Route planning\\
		\texttt{620.omnetpp} & Discrete Event simulation -- computer network\\
		\texttt{623.xalancbmk} & XML to HTML conversion via XSLT\\
		\texttt{625.x264} & Video compression\\
		\texttt{631.deepsjeng} & Artificial Intelligence: alpha-beta tree search (Chess)\\
		\texttt{641.leela} & Artificial Intelligence: Monte Carlo tree search (Go)\\
		\texttt{648.exchange2} & Artificial Intelligence: recursive solution generator (Sudoku)\\
		\texttt{657.xz} & General data compression\\
		\bottomrule
	\end{tabular}
	\caption[SPEC CPU 2017 workload description]%
	{A description of the workloads covered by \textit{SPEC CPU 2017}'s \texttt{intspeed} suite~\cite{spec-cpu-doc}.}
	\label{tab:spec-description}
\end{table}


Measuring the performance of the DBT was accomplished by using the tools in \textit{SPEC CPU 2017}'s \texttt{intspeed} suite of benchmarks.
This not only generates reproducible and widely accepted results in the industry, it also validates the results produced during the run, thus ruling out any errors in the benchmark's translation.

The \texttt{intspeed} suite also presents a variety of different workloads to the translator that are based on real-life scenarios, thus producing an accurate and understandable overview of the DBT's performance in a non-controlled environment.
An overview of the workloads covered by the aforementioned suite can be found in table~\ref{tab:spec-description}.
Further context is provided by performance testing using the data compression utility \textit{gzip}~\cite{gzip}, where compression time is compared between runs on a native machine, in QEMU and in the DBT\@.

All testing was performed on an x86--64 8-core \textit{Intel Xeon Bronze 3106} system clocked at $1,70$ GHz base with $78$ GiB of physical memory, running \textit{Ubuntu 18.04.3 LTS}, kernel version \textit{4.15.0-70-generic}.
The DBT was compiled via \texttt{CMAKE\_BUILD\_TYPE} set to \texttt{Release} and \texttt{CMAKE\_INTERPROCEDURAL\_OPTIMIZATION} enabled, which implies \texttt{-O3} and \texttt{-flto -fno-fat-lto-objects}.

The benchmarks were compiled using compiler optimisation level \texttt{-O3} and linked statically.
For the native run \texttt{-march=x86-64} was used on GCC version 7.5.0 from the Ubuntu standard repo.
The RISC--V binaries for our translator and QEMU were compiled using \texttt{-march=rv64ima} and \texttt{-mabi=lp64}.
A self compiled GCC with the sources taken from the official toolchain repo at version 10.1.0 was used.
This made it necessary to also specify \texttt{-fcommon -fallow-argument-mismatch} to stay fully compatible.

\subsubsection{Results}

% ======= SPEC CPU Results =======
% Results of the intspeed SPEC CPU 2017 runs.
% ================================
\pgfplotstableread[col sep=comma]{benchmarks/spec-base/native.csv}\nativetable
\pgfplotstableread[col sep=comma]{benchmarks/spec-base/dbt.csv}\dbttable
\pgfplotstablecreatecol[create col/copy column from table=\nativetable{Est. Base Run Time}]{native}\dbttable
\pgfplotstablecreatecol[create col/expr={\thisrow{Est. Base Run Time} / \thisrow{native}}]{ratio}\dbttable

\pgfplotstablesort[sort key = ratio, sort cmp=float >]\sorteddbttable{\dbttable}

\begin{figure}[h]
	\centering
	\begin{tikzpicture}
		\begin{axis}[%
			title = {SPEC CPU 2017 \texttt{intspeed} Results},
			ybar,
			area legend,
			ylabel = {Execution time [ratio to native]},
			xtick = data,
			xtick style = {draw = none},
			xticklabel style = {
				inner sep = 0pt,
				anchor = north east,
				rotate = 60
			},
			ytick = {1.0, 2.0, 3.0, 4.0, 5.0, 6.0},
			scaled y ticks = false,
			xticklabels from table={\sorteddbttable}{Benchmark},
			ymin = 0, ymax = 6.5,
			ymajorgrids = true,
			bar width = 5pt,
			height = 7.0cm,
			width = 0.9\linewidth,
			legend style = {
				at = {(0.98, 0.97)},
				anchor = north east,
				legend columns = 3,
				column sep = 0.2cm
			}
		]
			% Native results
			\pgfplotstableread[col sep=comma]{benchmarks/spec-base/native.csv}\nativelocaltable
			\pgfplotstablecreatecol[create col/copy column from table=\nativetable{Est. Base Run Time}]{native}\nativelocaltable
			\pgfplotstablecreatecol[create col/expr={\thisrow{Est. Base Run Time} / \thisrow{native}}]{ratio}\nativelocaltable
			\pgfplotstablesort[sort key = ratio, sort key from ={\dbttable}, sort cmp=float >]\nativelocalsortedtable{\nativelocaltable}
			\addplot+[fill=era-native, draw=black] table [x expr=\coordindex, y=ratio] \nativelocalsortedtable;

			% QEMU results
			\pgfplotstableread[col sep=comma]{benchmarks/spec-base/qemu.csv}\qemutable
			\pgfplotstablecreatecol[create col/copy column from table=\nativetable{Est. Base Run Time}]{native}\qemutable
			\pgfplotstablecreatecol[create col/expr={\thisrow{Est. Base Run Time} / \thisrow{native}}]{ratio}\qemutable
			\pgfplotstablesort[sort key = ratio, sort key from ={\dbttable}, sort cmp=float >]\qemusortedtable{\qemutable}
			\addplot+[fill=era-qemu, draw=black] table [x expr=\coordindex, y=ratio] \qemusortedtable;

			% DBT results
			\addplot+[fill=era-dbt-1, draw=black] table [x expr=\coordindex, y=ratio] \sorteddbttable;

			\legend{native, QEMU, DBT}
		\end{axis}
	\end{tikzpicture}
	\caption[SPEC CPU 2017 Results]%
	{Results of \texttt{ref}-workload runs of \textit{SPEC CPU 2017}'s \texttt{intspeed} (normalised, lower is better).}
	\label{fig:spec-results}
\end{figure}
% ================================

\subsubsection{Analysis}
Figure~\ref{fig:spec-results} shows normalized performance results of the \textit{SPEC CPU 2017} \texttt{intspeed} benchmarks, effectively showing how much overhead QEMU and our translator caused versus the same benchmark compiled and run natively.
Some of the overhead must of course be attributed to the architectural differences between x86 and RISC--V resulting in needing more instructions in RISC-V assembly than x86.
This means these results do not directly measure the overhead vs.\ native that the whole translator infrastructure (parsing, translation, code cache etc.) causes.
What we can compare though, is the relative results of QEMU and our translator, since both use the same compiler and thus get the same binary.
This means the results are a measure for the relative efficiency of the infrastructure and the quality of the generated code.

Through the various performance optimisations mentioned in section~\vref{sec:optimise} we are able to reach our goal of consistently outperforming QEMU\@.
In some cases the advantage is only slight, but in other workloads like the 602.gcc compiler benchmark the advantage grows to a comfortable $80\,\%$.

Most benchmarks show runtimes of about $1.9$x native with \texttt{625.x264} and \texttt{600.perlbench} being the outliers.

\texttt{625.x264} on x86 heavily takes advantage of vectorisation, which RISC-V does not yet support, meaning that the compiler will have to generate loops that run more often, thus needing significantly more instructions for the same result.
Retrospective vectorisation on the translator side is not easy since this would involve detecting the vectorisable loop patterns on assembler level, a task that even the compiler often isn't very effective at, even though it has the knowledge of the entire program.
There are also instances where it relies heavily on 32 bit integer arithmetic, which in RISC-V always causes the results to be sign extended to the 64 bit register width, in comparison to x86 which zero extends in these cases.
Thus many consecutive 32 bit instructions on the same values cause a lot of redundant sign extensions.
A future version of the translator could do the sign extensions lazily to save on a bunch of redundant work in some cases.

The \texttt{600.perlbench} on the other hand has a lot of conditional branches and jumps in the hot blocks.
This causes a lot of context switches since recursive translation currently is only employed for unconditional branches/calls.
Recursively translating the path that is considered hot by the compiler could improve performance by a bit.
It also potentially causes redundant work for the translator, since jumps to the middle of a basic block currently are handled by treating it like a new block beginning at the jump target.



\subsection{Evaluation of translator optimisations}
\begin{table}
	\centering
	\begin{tabular}{rl}
		\toprule
		\textbf{Option} & \textbf{Description}\\
		\midrule
		\texttt{no-ras} & Disable the return address stack\\
		\texttt{no-chain} & Disable block chaining\\
		\texttt{no-jump} & Disable recursive jump target translation\\
		\texttt{no-fusion} & Disable macro operation fusion\\
		\texttt{none} & All of the above\\
		\bottomrule
	\end{tabular}
	% state: 5afb8706e482917866a3507a54d1512befe56a21
	\caption[Translator optimisation options]%
	{The options for translator optimisations, as seen in \texttt{----optimize=help}.}
	\label{tab:opt-options}
\end{table}

In order to evaluate the optimisations built into the translator, we ran the \textit{SPEC CPU 2017} suite with various combinations of the available optimisation options in the same translator version (\texttt{v1.3.1}, the final release in the project's main development cycle).

The results of these runs can be seen in figure~\ref{fig:opt-compare}, and an overview of the switches specified in the figure's legend can be found in table~\ref{tab:opt-options}.

% ======= Optimisation comparison results =======
% Results of the optimisation option comparisons.
% Show all SPEC run results and analyse below.
% ===============================================
\pgfplotstableread[col sep=comma]{benchmarks/spec-compare/base.csv}\basetable
\pgfplotstableread[col sep=comma]{benchmarks/spec-compare/no-ras.csv}\norastable
\pgfplotstableread[col sep=comma]{benchmarks/spec-compare/no-fusion.csv}\nofusiontable
\pgfplotstableread[col sep=comma]{benchmarks/spec-compare/no-jump-no-ras.csv}\nojumpnorastable
\pgfplotstableread[col sep=comma]{benchmarks/spec-compare/none.csv}\nonetable

\pgfplotstablecreatecol[create col/copy column from table=\basetable{Est. Base Run Time}]{base}\basetable
\pgfplotstablecreatecol[create col/expr={\thisrow{Est. Base Run Time} / \thisrow{base}}]{ratio}\basetable

\pgfplotstablecreatecol[create col/copy column from table=\basetable{Est. Base Run Time}]{base}\norastable
\pgfplotstablecreatecol[create col/expr={\thisrow{Est. Base Run Time} / \thisrow{base}}]{ratio}\norastable

\pgfplotstablecreatecol[create col/copy column from table=\basetable{Est. Base Run Time}]{base}\nofusiontable
\pgfplotstablecreatecol[create col/expr={\thisrow{Est. Base Run Time} / \thisrow{base}}]{ratio}\nofusiontable

\pgfplotstablecreatecol[create col/copy column from table=\basetable{Est. Base Run Time}]{base}\nojumpnorastable
\pgfplotstablecreatecol[create col/expr={\thisrow{Est. Base Run Time} / \thisrow{base}}]{ratio}\nojumpnorastable

\pgfplotstablecreatecol[create col/copy column from table=\basetable{Est. Base Run Time}]{base}\nonetable
\pgfplotstablecreatecol[create col/expr={\thisrow{Est. Base Run Time} / \thisrow{base}}]{ratio}\nonetable

\begin{figure}[h]
	\centering
	\begin{tikzpicture}
		\begin{axis}[%
			title = {Optimisation option evaluation},
			ybar,
			area legend,
			ylabel = {Execution time [ratio to base]},
			xtick = data,
			xtick style = {draw = none},
			xticklabel style = {
				inner sep = 0pt,
				anchor = north east,
				rotate = 60
			},
			xticklabels from table={\basetable}{Benchmark},
			ytick = {0, 1, 2, 3, 4, 5, 6, 7, 8, 9, 10, 11},
			ymin =0, ymax = 11.5,
			ymajorgrids = true,
			bar width = 3pt,
			height = 9.0cm,
			width = 0.9\linewidth,
			legend style = {
				at = {(0.98, 0.97)},
				anchor = north east,
				legend columns = 3,
				column sep = 0.2cm
			}
		]
			% base results
			\addplot+[fill=era-native, draw=black] table [x expr=\coordindex, y=ratio] \basetable;
			
			% no-fusion results
			\addplot+[fill=era-dbt-1, draw=black] table [x expr=\coordindex, y=ratio] \nofusiontable;

			% no-ras results
			\addplot+[fill=era-qemu, draw=black] table [x expr=\coordindex, y=ratio] \norastable;
			
			% no-jump-no-ras results
			\addplot+[fill=era-dbt-2, draw=black] table [x expr=\coordindex, y=ratio] \nojumpnorastable;
			
			% none results
			\addplot+[fill=era-dbt-3, draw=black] table [x expr=\coordindex, y=ratio] \nonetable;

			\legend{base, \texttt{no-fusion}, \texttt{no-ras}, \texttt{no-jump}{,~} \texttt{no-ras}, \texttt{none}}
		\end{axis}
	\end{tikzpicture}
	\caption[Translator optimisation evaluation results]%
	{Results of \texttt{ref}-workload runs of \textit{SPEC CPU 2017}'s \texttt{intspeed} with various optimisation option combinations (normalised, lower is better).}
	\label{fig:opt-compare}
\end{figure}
% ================================


Macro operation fusion does not seem to provide a large performance benefit, in most benchmarks the numbers do not even suggest any performance increase above natural deviation of benchmark runs.
This means the implemented pattern matching does not give the desired effect of a good performance increase.
Further tweaking of the checked patterns might make this optimisation more worthwhile.

The return address stack provided for a significant advantage in some benchmarks.
Especially the function call heavy \texttt{620.omnetp}, \texttt{623.xalancbmk}, \texttt{631.deepsjeng}, \texttt{641.leela} benchmarks showed good performance gains of over $50\,\%$.
The \texttt{600.perlbench}, as well as the \texttt{648.exchange2} and \texttt{657.xz} benchmarks where most of the runtime is spent in only a couple loops naturally could not benefit a lot.

Recursive jump translation without also utilising the return address stack only provided a performance increase over disabling both in some benchmarks.
The main reason for this might be that this also makes context switches necessary on unconditional jumps that aren't function calls or returns.
This makes jump-heavy benchmarks take a performance hit while jump-light benchmarks are almost unaffected.

Expectedly, the highest performance penalty was incurred by disabling chaining as well.
This makes a context switch back to the translator necessary for every executed basic block.
The benchmarks that are less impacted by disabling block chaining are the ones where fewer basic blocks were executed relative to their runtime.
This correlates with the fact that the most executed blocks of these benchmarks contain more instructions and hence execute for a longer time.

Furthermore, the lazy replacement register handling described in section~\ref{sec:reg-handle} had a high impact in some benchmarks, most notably \texttt{657.xz}, providing for a roughly $45\,\%$ performance increase -- more than any other optimisation apart from block chaining.
Any workload that frequently accesses registers that are not statically mapped as per table~\vref{tab:static-register-mapping} benefits significantly from this style of register handling, as these registers will essentially behave as if they were statically mapped within the confines of that single basic block.

Of course, our chosen \textit{least recently used}-approach to register replacement into three temporary slots suffers from the same issue known from caching:
If a program accesses the same four not-statically-mapped registers in order in a loop, the algorithm will always replace and write back the value that would be needed next.
Preventing this issue, however, is not a trivial task even when presented with the entirety of the guest program.
So, as this approach can not perform worse than accessing the register file in memory for each instruction and has lead to significant performance increases in some workloads, this is a very worthwhile optimisation.



\subsection{Data compression via gzip}
% gzip results
Next to the results of the \textit{SPEC CPU 2017} suite, it is also valuable to measure the performance of the translator in real-world workloads by running data compression via \textit{gzip}.

For better comparability, both the native and RISC-V \textit{gzip} binaries were compiled manually with the compiler optimisation level \texttt{-O3} alongside the linker flag \texttt{-static}.
The RISC-V ABI was setup with \texttt{--march=rv64ima} and \texttt{--mabi=lp64}.

% ======= gzip execution time =======
% Execution time of compression (500 MB, 5 runs).
% ===================================
\begin{figure}[h]
	\centering
	\begin{tikzpicture}
		\begin{axis}[%
			title = {Random Data Compression (\textit{gzip}, 500 MB, 5 runs)},
			xbar,
			area legend,
			xlabel = {Execution time [ratio to native] (lower is better)},
			symbolic y coords = {Native,QEMU,DBT,\ldots unoptimised},
			xmin = 0,
			bar shift = 0.0cm,
			y dir = reverse,
			enlarge y limits = {value=0.2, auto},
			xmajorgrids = true,
			height = 5.5cm,
			width = 13.0cm,
			legend style = {
				at = {(0.5, -0.45)},
				anchor = north,
				legend columns = 4,
				column sep = 0.2cm
			}
		]
			\addplot+ [
				fill=era-native,
				draw=black
			] coordinates {
				(1.0,Native) % 44.15
			};

			\addplot+ [
				fill=era-qemu,
				draw=black
			] coordinates {
				(2.767157418,QEMU) % 122.17
			};
			
			\addplot+ [
				fill=era-dbt-1,
				draw=black
			] coordinates {
				(1.911438279,DBT) % 84.39
			};
			
			\addplot+ [
				fill=era-dbt-2,
				draw=black
			] coordinates {
				(9.527180068,\ldots unoptimised) % 420.625
			};
			
			\legend{Native, QEMU, DBT, DBT unoptimised}
		\end{axis}
	\end{tikzpicture}
	\caption[Execution time of gzip compression]%
	{Execution time of gzip file compression (500 MB of random data, 5 runs) in seconds (normalised, lower is better).\\Unoptimised run executed with \texttt{----optimize=none}.}
	\label{fig:gzip-execution-time}
\end{figure}
% ===================================

Figure~\vref{fig:gzip-execution-time} lists the execution times of \textit{gzip} compressing a pseudo-random $500$ MB file sourced from \texttt{/dev/urandom}\footnote{Reproducible via \texttt{base64 /dev/urandom | head -c 524288000 > random.txt;}}.

Through our very efficient return address stack, recursive jump target translation, macro operation fusion and, most importantly, block chaining we are able to significantly outperform QEMU in random data compression by nearly 45\,\%.
The achieved performance of approximately two times the execution time of a native run is in line with the \textit{SPEC CPU 2017} results shown in figure~\ref{fig:spec-results}.

As mentioned in the caption, the unoptimised run was performed with the command line option \texttt{----optimize=none}, which disables all of the optimisation features mentioned above.
The translator will then have to translate every block one-by-one, jump back into the main loop on every block end and fetch the next position based on the current program counter.






















